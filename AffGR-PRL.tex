\documentclass[aps,prl,twocolumn,superscriptaddress,showpacs,showkeys]{revtex4-1}

\usepackage{amsmath,amsthm,latexsym,amssymb,amsfonts}
\usepackage{xcolor}
\usepackage[%
  colorlinks=true,
  urlcolor=blue,
  linkcolor=blue,
  citecolor=blue
]{hyperref}
\usepackage{etoolbox}
\usepackage{breqn}

\makeatletter
\let\cat@comma@active\@empty
\makeatother

%------------------
%--------- Definitions
%------------------
/home/oscar/Documents/LatexFiles/Def-article.tex

\hypersetup{%
  pdftitle={Einstein's Gravity from an Affine Model},
  pdfauthor={Aureliano Skirzewski,}{Oscar Castillo-Felisola},
  pdfkeywords={Affine Gravity,} {Torsion,} {Generalised Gravity.},
  pdflang={English}
}


%------------------
%--------- Document
%------------------
\begin{document}

\title{Einstein's gravity from an affine model}


\author{Aureliano \surname{Skirzewski}}
\email{askirz@gmail.com}
\affiliation{\CFF.}

\author{Oscar \surname{Castillo-Felisola}}
\email{o.castillo.felisola@gmail.com}
\affiliation{\CCTVal.}
\affiliation{\UTFSM.}

%% --------- Abstract
\begin{abstract}
  We show that general Einstein manifolds (with or without cosmological constant) are a possible solution to the equations of motion of a recently formulated affine model of gravity, when the connection is the one of Levi-Civita. Moreover, we find an equivalence between the minimally coupled massless scalar field in General Relativity with a ``minimally'' coupled scalar field in this affine model.
\end{abstract}

\pacs{02.40.Ma,04.50.Kd,04.90.+e}
\keywords{Affine Gravity, Torsion, Generalised Gravity.}


\maketitle

%%%%%%%%% THE MODEL %%%%%%%%%
In a previous paper~\cite{Skirzewski:2014eta}, we presented a purely affine gravitational model in four dimensions built up entirely on the basis of full diffeomorphism invariance, and restricting ourselves to consider only na\"ive power-counting renormalizable terms. We showed that its non-relativistic limit around a homogeneous and isotropic spacetime yields a Newtonian gravity. We also highlighted the possibility of using standard methods to quantize the model, due to its polynomial structure (unlike the earliest proposals~\cite{Eddington1923math,schrodinger1950space}), and also the likelihood of avoiding the uniqueness of diffeomorphism-invariant states from Loop Quantum Gravity programme~\cite{Lewandowski:2005jk}. Additionally, it is well known that the three-point graviton vertices lead to non-renormalizability of the theory --- in a model independent way ---~\cite{McGady:2013sga,Camanho:2014apa}, notwithstanding in the considered framework all graviton self-interaction is mediated by the torsion which might allow to bypass the general postulates supporting the theorem.

The four-dimensional action is written in terms of the curvature and torsion of an affine connection, $\Gamma^\mu{}_{\rho\sigma}$, which accepts a decomposition on irreducible components as
\begin{equation}
  \hat{\Gamma}^\mu{}_{\rho\sigma} = {\Gamma}^\mu{}_{\rho\sigma} + T^\mu{}_{\rho\sigma} = {\Gamma}^\mu{}_{\rho\sigma} + \epsilon_{\rho\sigma\lambda\kappa}T^{\mu,\lambda\kappa}+A_{[\rho}\delta^\mu_{\nu]},
\end{equation}
where ${\Gamma}^\mu{}_{\rho\sigma}$ is symmetric in the lower indices, the $A_\mu$ is a vector field, the trace of torsion, and  $T^{\mu,\lambda\kappa}$ is a Curtright field~\cite{Curtright:1980yk}. Since no metric is present  the epsilon symbols are not related but instead we demand that \mbox{$\epsilon^{\delta\eta\lambda\kappa}\epsilon_{\mu\nu\rho\sigma}=4!\delta^{\delta}{}_{[\mu}\delta^\eta{}_{\nu}\delta^{\lambda}{}_{\rho} \delta^\kappa{}_{\sigma]}$.}
\begin{widetext}
  Using the above decomposition, the most general (presumably renormalizable) action preserving diffeomorphisms is~\footnote{We have dropped the contribution to a boundary term, the complete action is written in Ref.~\cite{Skirzewski:2014eta}.}
  \begin{dmath}
    \label{4dfull}
    S[{\Gamma},T,A] =
    \int\dn{4}{x}\Bigg[
      B_1\, R_{\mu\nu}{}^{\mu}{}_{\rho} T^{\nu,\alpha\beta}T^{\rho,\gamma\delta}\epsilon_{\alpha\beta\gamma\delta}
      +B_2\, R_{\mu\nu}{}^{\sigma}{}_\rho T^{\beta,\mu\nu}T^{\rho,\gamma\delta}\epsilon_{\sigma\beta\gamma\delta}
      +B_3\, R_{\mu\nu}{}^{\mu}{}_{\rho} T^{\nu,\rho\sigma}A_\sigma
      +B_4\, R_{\mu\nu}{}^{\sigma}{}_\rho T^{\rho,\mu\nu}A_\sigma
      +B_5\, R_{\mu\nu}{}^{\rho}{}_\rho T^{\sigma,\mu\nu}A_\sigma
      +C_1\, R_{\mu\rho}{}^{\mu}{}_\nu \nabla_\sigma T^{\nu,\rho\sigma}
      +C_2\, R_{\mu\nu}{}^{\rho}{}_\rho \nabla_\sigma T^{\sigma,\mu\nu} 
      +D_1\, T^{\alpha,\mu\nu}T^{\beta,\rho\sigma}\nabla_\gamma T^{(\lambda, \kappa) \gamma}\epsilon_{\beta\mu\nu\lambda}\epsilon_{\alpha\rho\sigma\kappa}
      +D_2\,T^{\alpha,\mu\nu}T^{\lambda,\beta\gamma}\nabla_\lambda T^{\delta,\rho\sigma}\epsilon_{\alpha\beta\gamma\delta}\epsilon_{\mu\nu\rho\sigma}
      +D_3\,T^{\mu,\alpha\beta}T^{\lambda,\nu\gamma}\nabla_\lambda T^{\delta,\rho\sigma}\epsilon_{\alpha\beta\gamma\delta}\epsilon_{\mu\nu\rho\sigma}
      +D_4\,T^{\lambda,\mu\nu}T^{\kappa,\rho\sigma}\nabla_{(\lambda} A_{\kappa)} \epsilon_{\mu\nu\rho\sigma}
      +D_5\,T^{\lambda,\mu\nu}\nabla_{[\lambda}T^{\kappa,\rho\sigma} A_{\kappa]} \epsilon_{\mu\nu\rho\sigma}
      +D_6\,T^{\lambda,\mu\nu}A_\nu\nabla_{(\lambda} A_{\mu)}
      +D_7\,T^{\lambda,\mu\nu}A_\lambda\nabla_{[\mu} A_{\nu]} 
      +E_1\,\nabla_{(\rho} T^{\rho,\mu\nu}\nabla_{\sigma)} T^{\sigma,\lambda\kappa}\epsilon_{\mu\nu\lambda\kappa}
      +E_2\,\nabla_{(\lambda} T^{\lambda,\mu\nu}\nabla_{\mu)} A_\nu
      +T^{\alpha,\beta\gamma}T^{\delta,\eta\kappa}T^{\lambda,\mu\nu}T^{\rho,\sigma\tau}
      \Big(F_1\,\epsilon_{\beta\gamma\eta\kappa}\epsilon_{\alpha\rho\mu\nu}\epsilon_{\delta\lambda\sigma\tau}
      +F_2\,\epsilon_{\beta\lambda\eta\kappa}\epsilon_{\gamma\rho\mu\nu}\epsilon_{\alpha\delta\sigma\tau}\Big) 
      +F_3\, T^{\rho,\alpha\beta}T^{\gamma,\mu\nu}T^{\lambda,\sigma\tau}A_\tau \epsilon_{\alpha\beta\gamma\lambda}\epsilon_{\mu\nu\rho\sigma}
      +F_4\,T^{\eta,\alpha\beta}T^{\kappa,\gamma\delta}A_\eta A_\kappa\epsilon_{\alpha\beta\gamma\delta}\Bigg].
  \end{dmath}
\end{widetext}
Although there is no obvious equivalence between the action in Eq.~\eqref{4dfull} and General Relativity (GR), particularly due to the lack of a source metric field, we analysed in Ref.~\cite{Skirzewski:2014eta} the scalar perturbations around a static homogeneous and isotropic background, and in the non-relativistic (low-momentum) limit, the geodesic deviation of inertial observers is also discussed.

Additionally, we included perturbative inhomogeneous sources to the connection field equations, by considering a generic matter action. Albeit the inclusion of matter in a non-metric spacetime is a nonstandard procedure, we used ``Eddington's metric density''%% \hl{JZ,AA: satisfy metric conditions? e.g. non-deg.}
defined as the variation of the action with respect to the symmetric part of the Ricci tensor~(see Refs.~\cite{Eddington1923math,schrodinger1950space,Poplawski:2012bw}), i.e.,%% \hl{JZ: def. $R_{(\mu\nu)}$}
\begin{equation}
  \label{metric}
  \frac{\delta\ }{\delta R_{(\mu\nu)}} S[\Gamma] \equiv \sqrt{\mathfrak{g}} \, \mathfrak{g}^{\mu\nu} = \bar{g}^{\mu\nu} ,
\end{equation}
by assuming a matter action that depends on $\bar{g}^{\mu\nu}$.

From the geodesic equations and the linearized action on scalar perturbation, we concluded that the effective potential
\begin{equation}
  \label{NewtonPot}
  v_L = \frac{1}{8\pi} \frac{ \partial\mathcal{L}_{\text{Matter}} }{ \partial \bar{g}^{00} } \frac{1}{|\vec{x}|},
\end{equation}
is the usual Newtonian potential for a massive source.

%%%%%%%%% COMPARISON GR %%%%%%%%%
Although we obtained the Newtonian gravitational potential~\eqref{NewtonPot}, the post-Newtonian approximation is necessary to explain gravitational phenomena like Mercury's perihelion and the deviation of light by a gravitational source. Therefore, we should go beyond the Newtonian limit in order to compare our model with GR.

%%\hl{A comparison between our affine model and GR requires the imposition of vanishing torsion (JZ: Why?).}
Even though a comparison between our affine model and GR does not require the imposition of vanishing torsion, at this stage we shall focus on a sector of the theory in which the connection is metric and torsion-free. Interestingly, the field equations from Eq.~\eqref{4dfull} can be consistently ``truncated'' under such requirements, and the only nontrivial equation is the one for the Curtright, $T^{\lambda,\mu\nu}$,
\begin{equation*}
  \kappa_1 \nab{[\rho} R_{\mu]\lambda}{}^{\lambda}{}_{\nu} + \kappa_2 \nab{\nu} R_{\mu\rho}{}^\lambda{}_\lambda = 0.
\end{equation*}
In the considered sector the connection is compatible with a metric, therefore, the second term vanishes and the gravitational equations are
\begin{equation}
  \nab{[\rho} R_{\mu]\nu} = 0.
  \label{SimpleEOM}
\end{equation}

The above equation is a generalization of a condition known in Riemannian geometry as covariantly constant Ricci curvature --- aka parallel Ricci curvature ---, \mbox{$\nab{\rho} R_{\mu\nu} = 0$.} The parallel Ricci curvature is a generalization of the Einstein condition which assures that although the manifold is not generally Einstein, the metric has to be locally a product of Einstein metrics~\cite{Besse}.

Interestingly, a corollary is that all Einstein manifolds, \mbox{$R_{\mu\nu} \propto g_{\mu\nu}$,} satisfy the parallel Ricci condition, and therefore every vacuum solution to the Einstein's equations solves the (torsion-free) field equations of our model. Consequently, the fact that the non-relativistic limit of the gravitational potential in Eq.~\eqref{NewtonPot} yields a Newtonian potential seems clearer, and we can argue that even the post-Newtonian corrections coming from GR are present in the chosen scenario of our model.

Additionally, for a Riemannian manifold $(\Mi,g)$ the condition in Eq.~\eqref{SimpleEOM} is equivalent to the following statements (see Ref.~\cite{Derdzinski:1985,Besse}): (i) the Ricci tensor is a Codazzi tensor~\footnote{A Codazzi tensor is a symmetric $(0,2)$-type tensor, $T$, satisfying the condition \mbox{$D_X T(Y,Z) = D_Y T(X,Z)$.}}, (ii) the manifolds has harmonic curvature, i.e. \mbox{$\nabla^\mu R_{\mu\nu}{}^\lambda{}_\rho = 0$,} and (iii) in the four-dimensional case the manifold has harmonic Weyl tensor~\cite{Berger:1969} and constant scalar curvature.

%%%%%%%%% MATTER %%%%%%%%%
Until now, the most general diffeomorphism-invariant and power-counting renormalizable (gravitational) theory for an affine connection has been built, and we have showed that in a certain sector it is equivalent to GR. In what follows, we show an attempt of including scalar matter on the model. The theory does not require a spacetime metric, but instead we can use the Eddington's inverse metric density ($\bar{g}^{\mu\nu}$), arising as the functional variation of the action with respect to the symmetric part of the Ricci tensor, to contract indices and build Lagrangian densities for the matter content. Following our precept, the matter content --- just scalar matter --- should couple to $\bar{g}^{\mu\nu}$, which for Eq.~\eqref{4dfull} is
\begin{dmath}
  \bar{g}^{\mu\nu} = B_1\, \epsilon_{\lambda\kappa\rho\sigma} T^{\mu, \lambda\kappa} T^{\nu, \rho\sigma} + B_3\, A_\lambda T^{\mu,\nu\lambda} + C_1\, \nabla_\lambda T^{\mu,\nu\lambda}.
\end{dmath}
We consider the action provided by the ``kinetic term''
\begin{dmath}
  \label{ScalarAction}
  S_\phi = - \alpha \int \dn{4}{x} \Big( C_1\, \nabla_\lambda T^{\mu,\nu\lambda}  + B_3\, A_\lambda T^{\mu,\nu\lambda} + B_1\, \epsilon_{\lambda\kappa\rho\sigma} T^{\mu, \lambda\kappa} T^{\nu, \rho\sigma} \Big) \partial_\mu\phi\partial_\nu\phi,
\end{dmath}
which makes a nontrivial contribution to the field equations once we restrict to the sector of interest, where the torsion is set to zero and the connection is metric.

The equation for the Curtright field when the scalar field is turned on is
\begin{equation*}
  \nabla_{[\sigma} R_{\rho]\mu}{}^{\mu}{}_\nu - \frac{C_2}{C_1} \nabla_\nu  R_{\rho\sigma}{}^{\mu}{}_\mu - \alpha \nabla_{[\sigma} \Big( \partial_{\rho]}\phi \partial_\nu\phi \Big) = 0,
\end{equation*}
which under our considerations simplifies to 
\begin{equation}
  \nabla_{[\sigma} R_{\rho]\nu} - \alpha \nabla_{[\sigma} \Big( \partial_{\rho]}\phi \partial_\nu\phi \Big) = 0.
  \label{SimpleEOMwS}
\end{equation}
%% when the connection is the Levi-Civita one.
In that case, we find a particular solution of Eq.~\eqref{SimpleEOMwS},%% \hl{JZ: What if non-LC connection?}
\begin{equation*}
  R_{\mu\nu} - \alpha \pa{\mu} \phi \pa{\nu} \phi = \Lambda g_{\mu\nu},
\end{equation*}
that can be written in the more conventional form
\begin{equation}
  R_{\mu\nu} - \frac{1}{2} g_{\mu\nu} R + \Lambda g_{\mu\nu} = \alpha \Big( \pa{\mu} \phi \pa{\nu} \phi - \frac{1}{2} g_{\mu\nu} \big( \partial\phi \big)^2 \Big).
\end{equation}
Additionally, the second Bianchi identity imposes
\begin{equation}
  \nabla^\mu \pa{\mu} \phi = 0.
\end{equation}
%% \hl{JZ: Esto muestra que la teoria contiene a R.G. com ocaso particular, pero tiene probablemente mucho mas. Que es lo extra?}
This condition is, in the sense argued in Ref.~\cite{Bekenstein:2014uwa}, the equation of motion for the scalar field. Notice that, the Euler--Lagrange equation of motion for the scalar field yields no information after taking the vanishing torsion ``truncation''.

It is well-known that this system of equations can be the obtained effectively from the Einstein--Hilbert action coupled minimally to a massless scalar field
\begin{equation}
  S_{\text{eff}} = \int \dn{4}{x} \sqrt{g} \Big( R + 2 \Lambda - \frac{1}{2} g^{\mu\nu} \pa{\mu}\phi \pa{\nu} \phi \Big).
\end{equation}

Summarising, we have shown that the  \emph{purely affine gravity} constructed in Ref.~\cite{Skirzewski:2014eta} possesses a well-defined torsionless limit, which generalizes the equations of motion  from GR. Consequently, all gravitational effects described by standard General Relativity is contained in our model, even though the torsion field is kept to zero. Regarding this aspect, we would like to emphasise that there is enough liberty to start considering new cosmological effects coming from both nonvanishing torsion and non-metric connections with vanishing torsion.
Moreover, our gravitational model coupled to a scalar field through the \emph{Eddington's metric} --- in the vanishing torsion consideration ---, is equivalent to GR minimally coupled to a massless scalar field. In a forthcoming paper, we will consider more general matter content, and some properties of the field equations which differ from the ones of GR.%%\hl{JZ: Ok, so?}

%%%%%%%%% ACKNOWLEGMENTS %%%%%%%%%
\begin{acknowledgments}
  We thank to A. Melfo and R. L. Bryant for their helpful discussions and inspiring comments, to J. Zanelli for his suggestions on the physical insides of the problem and careful but critical review of the manuscript, and also to K. Peeters for helpful advises in the manipulation of the software \textsc{Cadabra}~\cite{Peeters:2007wn,*peeters2007symbolic,*Peeters2007550}, which was used extensively to achieve the results presented in this paper.
  This work was partially supported by CONICYT (Chile) under project No. 79140040.
\end{acknowledgments}

%%%%%%%%% BIBLIOGRAPHY %%%%%%%%%
\bibliographystyle{apsrev4-1}
\bibliography{References.bib}

\end{document}

