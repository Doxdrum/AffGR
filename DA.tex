\section{\label{DA}Dimensional analysis for building the polynomial affine gravity}

We built the model using six ingredients, a Curtright ($T^{\mu,\nu\lambda}$), a vector ($A_\mu$), the covariant derivative defined with the Levi-Civita connection ($\nab{\mu}$), both Levi-Civita tensors ($\epsilon_{\mu\nu\lambda\rho}$ and $\epsilon^{\mu\nu\lambda\rho}$), and the Riemannian curvature ($R_{\mu\nu}{}^\lambda{}_\rho$).
Since the Riemannian curvature is defined as the commutator of the covariant derivative, it is not an independent field, so it will be left out of the analysis, and only five ingredients remain.

Our interest is in general to build tensor densities. Therefore, we need to account for the number of \emph{free} indices, and weight density of these quantities. Denote by $N(\Phi)$  the operator which count the number (and position) of indices of the field $\Phi$, being positive (negative) for upper (lower) indices. Thus, we have that
\begin{equation*}
  \begin{aligned}
    N( T^{\mu,\nu\lambda} ) &= 3 & N( A_\mu ) &= -1 & N(  \nab{\mu} ) &= -1 \\
    N( \epsilon_{\mu\nu\lambda\rho} ) &= -4 & & & N( \epsilon^{\mu\nu\lambda\rho} ) &= 4 
  \end{aligned}
\end{equation*}
Then, for a general expression of the form
\begin{equation*}
  T^a A^b \nabla^c {\epsilon_{\dots}}^d {\epsilon^{\dots}}^e,
\end{equation*}
the indices counting yield
\begin{equation}
  N( T^a A^b \nabla^c {\epsilon_{\dots}}^d {\epsilon^{\dots}}^e ) = n,
\end{equation}
with
\begin{equation}
  n = 3a -b -c -4d + 4e = 3 a -b - c + 4 \ell,
  \label{ni}
\end{equation}
where we defined $\ell = e - d$, for the sake of simplicity.

In the same spirit, we define an operator which counts the weight density,
\begin{equation*}
  \begin{aligned}
    W( T^{\mu,\nu\lambda} ) &= 1 & W( A_\mu ) &= 0 & W(  \nab{\mu} ) &= 0 \\
    W( \epsilon_{\mu\nu\lambda\rho} ) &= -1 & & & W( \epsilon^{\mu\nu\lambda\rho} ) &= 1 .
  \end{aligned}
\end{equation*}
Thus, the weight ($w$) of the general expression above is given by
\begin{equation}
  w = a + \ell.
  \label{wd}
\end{equation}

\subsection{\label{sec:im}Inverse metric density}

Now, we illustrate the usefulness of the \emph{dimensional analysis} by building the most general symmetric $\binom{2}{0}$-tensor density, which we call the \emph{inverse metric density}. This particular case fixes $n = 2$ and $w = 1$.


Equation~\eqref{wd} can solved by choosing either $a=1$ or $\ell = 1$, which imply that Eq.~\eqref{ni} is restricted to
\begin{equation*}
  3 - b - c = 2,
\end{equation*}
or
\begin{equation*}
  4 - b - c = 2,
\end{equation*}
respectively. The former, yields the terms
\begin{equation}
  T^{\mu,\nu\lambda} A_\lambda \text{ and } \nabla_\lambda T^{\mu,\nu\lambda},
\end{equation}
no other contraction of indices is allowed due to the symmetry. The latter, yields another possibility,
\begin{equation}
  \epsilon_{\lambda\kappa\rho\sigma} T^{\mu, \lambda\kappa} T^{\nu, \rho\sigma}.
\end{equation}

Finally, one can check that any other choice to solve Eq.~\eqref{wd} ---by allowing negative values of $\ell$--- make impossible to solve Eq.~\eqref{ni}. Ergo, there is no other term in a symmetric $\binom{2}{0}$-tensor density built up with these fields. Explicitly, this general tensor density is written in Eq.~\eqref{geng}. The summary of this analysis is presented in Table~\ref{tab:imd}.
\begin{table}
  \caption{Possible terms contributing to the inverse density metric. }
  \label{tab:imd}
  \begin{tabular}{|C{.23\linewidth}C{.23\linewidth}C{.23\linewidth}C{.23\linewidth}|}
    \hline
    $a$ & $b$ & $c$ & $\ell$ \\
    \hline
    1 & 1 & 0 & 0 \\
    1 & 0 & 1 & 0 \\
    2 & 0 & 0 &-1 \\
    0 & 2 & 0 & 1 \\
    0 & 1 & 1 & 1 \\
    0 & 0 & 2 & 1 \\
    \hline
  \end{tabular}
\end{table}

\subsection{\label{sec:ld}Lagrangian density}

\begin{table}
  \caption{Possible terms contributing to the Lagrangian density.}
  \label{tab:ld}
  \begin{tabular}{|C{.23\linewidth}C{.23\linewidth}C{.23\linewidth}C{.23\linewidth}|}
    \hline
    $a$ & $b$ & $c$ & $\ell$ \\
    \hline
    1 & 3 & 0 & 0\\
    1 & 2 & 1 & 0\\
    1 & 1 & 2 & 0\\
    1 & 0 & 3 & 0\\
    2 & 2 & 0 & -1\\
    2 & 1 & 1 & -1\\
    2 & 0 & 2 & -1\\
    3 & 1 & 0 & -2\\
    3 & 0 & 1 & -2\\
    4 & 0 & 0 & -3\\
    0 & 4 & 0 & 1\\
    0 & 3 & 1 & 1\\
    0 & 2 & 2 & 1\\
    0 & 1 & 3 & 1\\
    0 & 0 & 4 & 1\\
    \hline
  \end{tabular}
\end{table}

The work of building up the most general scalar density with our ingredients is as simple as before, but the analysis is much longer. Thus, we will show the procedure with lesser details than before.

First, we are interested in a scalar density, which sets $n = 0$ and $w = 1$ in Eqs.~\eqref{ni} and~\eqref{wd}, i.e.,
\begin{align}
  3a -b -c  + 4\ell &= 0, \label{nib}\\
  a + \ell &= 1. \label{wdb}
\end{align}

A possible solution of Eq~\eqref{wdb} is $a=1$ and $\ell = 0$. Such choice allows four possible solutions of Eq.~\eqref{nib}, but only three of them are nonvanishing. These generate the terms of Eq.~\eqref{4dfull} whose coefficients were called $B_3$, $B_4$, $B_5$, $C_1$, $C_2$, $D_6$, $D_7$ and $E_2$.

If one choose $\ell = 1$ for solving Eq.~\eqref{wdb}, all possible solutions for Eq.~\eqref{nib} are either vanishing or yield topological terms.
%%The action in Eq.~\eqref{4dfull} was written up to topological and boundary terms, thus 

The case with $a=2$ and $\ell = -1$ gives three possibilities to solve Eq.~\eqref{nib}. These choices give the terms the Lagrangian density whose coefficients are $B_1$, $B_2$, $D_4$, $D_5$, $E_1$, and $F_4$.

For $a = 3$ and $\ell = -2$ there are only two possible solutions of Eq.~\eqref{nib}, and these yield the terms of the action with coefficients $D_1$, $D_2$, $D_3$ and $F_3$. And also, the choice $a = 4$ and $\ell = -3$ solves the equations, and yield the terms in the action with coefficients $F_1$ and $F_2$.

Notice that for values $a \ge 5$, it is not possible to solve both Eqs.~\eqref{nib} and~\eqref{wdb} simultaneously. Therefore, we conclude that Eq.~\eqref{4dfull} is the most general action built up with these fields. A summary of the choices for building the Lagrangian density is presented in Table~\ref{tab:ld}.

